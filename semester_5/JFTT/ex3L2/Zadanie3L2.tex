\documentclass{article}
\usepackage[T1]{fontenc}
\usepackage[utf8]{inputenc}
\usepackage[polish]{babel}
\usepackage{amssymb}
\title{Zadanie 3 Lista 2}
\author{Paweł Grzegory}
\date{\today}
\begin{document}
\maketitle

\section{Zadanie}
Udowodnij, że język
\[
L = \{\,x \in \{0,1\}^* \land |x|_0 \le |x|_1 \le 2|x|_0\,\}
\]
nie jest regularny.

\section{Rozwiązanie}
Załóżmy nie wprost, że \(L\) jest regularny. Niech \(n\) będzie stałą z Lemmatu o pompowaniu. Weźmy słowo
\[
z = 0^n1^n,
\]
które należy do \(L\), ponieważ \(n = |z|_0 \le |z|_1 = n \le 2n = 2|z|_0\). \\
Z lematu o pompowaniu istnieje rozkład \(z = uvw\) taki, że \(|uv| \le n\) oraz \(|v| \ge 1\), i dla każdego \(i \ge 0\) mamy \(uv^iw \in L\). \\
\(|uv| \le n\), zatem fragment \(v\) składa się wyłącznie z zer, więc \(v = 0^k\) dla pewnego \(k \ge 1\).
\\
Dla \(i=2\) otrzymujemy
\[
uv^2w = uvvw = 0^{n+k}1^n =: a.
\]
W słowie \(a\) liczba zer wynosi \(|a|_0 = n+k\), a jedynek \(|a|_1 = n\). Ponieważ \(n+k > n\), zachodzi \(|a|_0 > |a|_1\), czyli nie spełniony jest warunek \(|x|_0 \le |x|_1\). Stąd \(uv^2w \notin L\), co przeczy założeniu, że wszystkie \(uv^iw\) należą do \(L\).
\\
Otrzymaliśmy sprzeczność, więc język \(L\) nie jest regularny. \(\square\)


\end{document}