\documentclass[a4paper,12pt]{article}
\usepackage[utf8]{inputenc}
\usepackage{polski}
\usepackage{amsmath}
\usepackage{graphicx}
\usepackage{listings}
\usepackage{geometry}
\geometry{margin=2.5cm}

\title{Sprawozdanie z Laboratorium \\ Obliczenia Numeryczne}
\author{Paweł Grzegory}
\date{\today}

\begin{document}

\maketitle

\section{Zadanie 1}
 
\subsection{Wyznaczanie epsilona maszynowego}

Celem zadania jest iteracyjne wyznaczenie epsilona maszynowego (macheps).

Epsilonem maszynowym (macheps) nazywamy najmniejszą dodatnią liczbę $\epsilon$ taką, że $\mathrm{fl}(1+\epsilon)>1$, gdzie $\mathrm{fl}$ oznacza operację zaokrąglenia w arytmetyce zmiennoprzecinkowej. Innymi słowy, macheps to odległość 1.0 od następnej reprezentowalnej liczby większej od 1.

Wyniki uzyskane w języku Julia:

\begin{center}
\begin{tabular}{l c c c}
\hline
Typ danych & Obliczony epsilon & \texttt{eps()} z Julii & Różnica \\
\hline
\texttt{Float16} & 9.765625000000000e-04 & 9.765625000000000e-04 & 0.000000000000000e+00 \\
\texttt{Float32} & 1.192092895507812e-07 & 1.192092895507812e-07 & 0.000000000000000e+00 \\
\texttt{Float64} & 2.220446049250313e-16 & 2.220446049250313e-16 & 0.000000000000000e+00 \\
\hline
\end{tabular}
\end{center}

Dane zawarte w plku nagłówkowym float.h

\begin{flushleft}
\begin{tabular}{@{}l c@{}}
\hline
Typ danych & Epsilon \\
\hline
\texttt{float} & 1.192093e-07 \\
\texttt{double} & 2.220446e-16 \\
\texttt{long double} & 1.084202e-19 \\
\hline
\end{tabular}
\end{flushleft}


\section{Cel ćwiczenia}
Celem ćwiczenia było porównanie wartości epsilona maszynowego obliczonego iteracyjnie z wartością zwracaną przez funkcję \texttt{eps()} w Julii.

\end{document}