\documentclass{article}
\usepackage[utf8]{inputenc}
\usepackage[T1]{fontenc}
\usepackage{polski}
\usepackage{amsmath}
\usepackage{amssymb}
\usepackage{amsthm}
\usepackage{geometry}
\usepackage{graphicx}
\geometry{a4paper, margin=1in}

\title{Sprawozdanie: Obliczenia Naukowe Lista 3}
\author{Paweł Grzegory}
\date{\today}

\begin{document}

\maketitle

\textbf{Oznaczenia:} $r$ - przybliżony pierwiastek, $v$ - wartość funkcji w punkcie $r$, $it$ - liczba iteracji, $err$ - kod błędu (0 - zbieżność osiągnięta, 1 - przekroczono maksymalną liczbę iteracji, 2 - pochodna bliska zeru).

\section{Zadania 1-3}
Zaimplementowano trzy metody numeryczne rozwiązywania równań nieliniowych $f(x)=0$ w module \texttt{functions.jl}.

\subsection*{Implementacja Funkcji (\texttt{functions.jl})}
\begin{enumerate}
    \item \textbf{Metoda Bisekcji (\texttt{bisection\_method})}
    \begin{itemize}
        \item Wymaga podania przedziału $[a,b]$, w którym funkcja $f(x)$ zmienia znak.
        \item Warunek stopu opiera się na długości przedziału (\texttt{delta}) lub na wartości funkcji w środku przedziału (\texttt{epsilon}).
        \item Zwraca kod błędu \texttt{err=1}, jeśli $f(a)\cdot f(b)\geq 0$.
    \end{itemize}

    \item \textbf{Metoda Newtona (\texttt{newton\_method})}
    \begin{itemize}
        \item Wymaga definicji funkcji $f(x)$ oraz jej pochodnej $f'(x)$.
        \item Zatrzymuje się i zwraca \texttt{err=2}, jeśli wartość pochodnej w bieżącym przybliżeniu spełnia $|f'(x_k)|<10^{-12}$(jest blikska zeru), co uniemożliwia dalsze iteracje.
        \item Iteracja: $x_{k+1}=x_k-\dfrac{f(x_k)}{f'(x_k)}$.
    \end{itemize}

    \item \textbf{Metoda Siecznych (\texttt{secant\_method})}
    \begin{itemize}
        \item Aproksymuje pochodną używając dwóch ostatnich przybliżeń, więc nie wymaga jawnej postaci $f'(x)$.
        \item Zwraca \texttt{err=1}, jeśli różnica wartości funkcji w mianowniku jest blikska zeru, co uniemożliwia obliczenie siecznej.
        \item Iteracja: $x_{k+1}=x_k - f(x_k)\dfrac{x_k-x_{k-1}}{f(x_k)-f(x_{k-1})}$.
    \end{itemize}
\end{enumerate}

\subsection*{Przeprowadzone Testy (\texttt{tests.jl})}
\begin{itemize}
    \item \textbf{Testy Bisekcji:} Potwierdzono zbieżność oraz poprawne zgłaszanie błędu \texttt{err=1} w przypadku braku zmiany znaku na końcach przedziału.
    \item \textbf{Testy Newtona:} Sprawdzono zbieżność oraz poprawne zwracanie \texttt{err=2} gdy pochodna jest bliska zeru (np. dla $f(x)=x^3+1$ przy $x_0=0.0$). Przetestowano też warunek przekroczenia maksymalnej liczby iteracji (\texttt{err=1}).
    \item \textbf{Testy Siecznych:} Potwierdzono zbieżność dla typowych przypadków oraz poprawne zgłaszanie błędu \texttt{err=1} przy problemach z obliczeniem siecznej lub przekroczeniu maksymalnej liczby iteracji.
\end{itemize}
\section{Zadanie 4: $\sin x - (\frac{1}{2} x)^2 = 0$}

Równanie: $f(x) = \sin(x) - (0.5 x)^2$. Dokładności: $\delta = 0.5 \cdot 10^{-5}$, $\epsilon = 0.5 \cdot 10^{-5}$.

\subsection*{Wyniki}
\begin{table}[h!]
\centering
\begin{tabular}{@{} l l l l c c @{}}
\hline
Metoda & Parametry & r & v & it & err \\
\hline
Bisekcja & [1.5, 2] & \texttt{1.9337539672851562} & \texttt{-2.7027680138402843e-7} & 16 & 0 \\
Newton & x0 = 1.5 & \texttt{1.933753779789742} & \texttt{-2.2423316314856834e-8} & 4 & 0 \\
Sieczne & x0 = 1, x1 = 2 & \texttt{1.933753644474301} & \texttt{1.564525129449379e-7} & 4 & 0 \\
\hline
\end{tabular}
\caption{Wyniki dla równania $\sin x - (0.5 x)^2 = 0$.}
\label{tab:zad4}
\end{table}

\subsection*{Interpretacja oraz Wnioski}
Wszystkie metody zbiegły do pierwiastka $r \approx 1.933754$.\textbf{Metoda Newtona} (4 iteracje) i \textbf{Metoda Siecznych} (4 iteracje) wykazały zbieżność znacznie szybszą niż \textbf{Bisekcja} (16 iteracji), co jest zgodne z teorią (zbieżność kwadratowa vs. liniowa).

\section{Zadanie 5: Znalezienie punktów przecięcia \(y = 3x\) i \(y = e^x\) metodą bisekcji}

Równanie: $f(x) = 3x - \exp(x)$ odpowiada punktom przecięcia wymienionych funkcji. Dokładności: $\delta = 10^{-4}$, $\epsilon = 10^{-4}$.

\begin{figure}[h]
\centering
\includegraphics[width=0.75\textwidth]{image.png}
\caption{Wykres funkcji $f(x)=3x -\exp(x)$}
\label{fig:zad5_plot}
\end{figure}

\begin{table}[h]
\centering
\begin{tabular}{@{} l l l l c c @{}}
\hline
Pierwiastek & Przedział & x & f(x) & iteracje & err \\
\hline
Pierwszy & [0, 1] & \texttt{0.619140625} & \texttt{9.066320343276146e-5} & 9 & 0 \\
Drugi & [1, 2] & \texttt{1.5120849609375} & \texttt{7.618578602741621e-5} & 13 & 0 \\
\hline
\end{tabular}
\caption{Wyniki dla równania $3x - e^{x} = 0$.}
\label{tab:zad5}
\end{table}

Z powodu skończonej prezycji arytmetyki zmiennopozycyjnej, otrzymane wyniki nie są dokładne, więc wartości funkcji f1 i f2 minimalnie się różnią.
\subsection*{Interpretacja}
     Metoda Bisekcji poprawnie znalazła oba pierwiastki.

\section{Zadanie 6: Porównanie Metod dla $f_1(x) = e^{1-x} - 1$ i $f_2(x) = x e^{-x}$}

Dokładności: $\delta = 10^{-5}$, $\epsilon = 10^{-5}$.

\subsection*{Funkcja $f_1(x) = e^{1-x} - 1$ (Pierwiastek $r=1$)}

\begin{center}
\begin{tabular}{@{} l l l l c c @{}}
\hline
Metoda & Parametry & r & v & it & err \\
\hline
Bisekcja & [0, 2] & \texttt{1.0} & \texttt{0.0} & 1 & 0 \\
Newton & x0 = 0.5 & \texttt{0.9999999998878352} & \texttt{1.1216494399945987e-10} & 4 & 0 \\
Sieczne & x0 = 0, x1 = 2 & \texttt{1.0000017597132702} & \texttt{-1.7597117218937086e-6} & 6 & 0 \\
\hline
\end{tabular}
\end{center}

\medskip
\textbf{Tabela:} Wyniki dla $f_1(x)=e^{1-x}-1$.

\subsection{Interpretacja dla $f_1(x)$:}
\begin{itemize}
    \item \textbf{Metoda Bisekcji}: Zbiegła w 1 iteracji do dokładnego pierwiastka $r=1.0$, ponieważ środek przedziału [0, 2] jest pierwiastkiem.
    \item \textbf{Metoda Newtona} z $x_0=0.5$: Zbiegła w 4 iteracjach.
    \item \textbf{Metoda Siecznych} z $x_0=0$, $x_1=2$: Zbiegła w 6 iteracjach, wolniej niż Newton.
\end{itemize}

\subsection*{Funkcja $f_2(x) = x e^{-x}$ (Pierwiastek $r=0$)}


\begin{table}[h]
\centering
\begin{tabular}{@{} l l l l c c @{}}
\hline
Metoda & Parametry & r & v & it & err \\
\hline
Bisekcja & [-1, 1] & \texttt{0.0} & \texttt{0.0} & 1 & 0 \\
Newton & x0 = -1.0 & \texttt{-3.0642493416461764e-7} & \texttt{-3.0642502806087233e-7} & 5 & 0 \\
Sieczne & x0 = -1.0, x1 = 1 & \texttt{-2.202990601857084e-6} & \texttt{-2.2029954550300214e-6} & 17 & 0 \\
\hline
\end{tabular}
\caption{Wyniki dla $f_2(x)=x e^{-x}$.}
\label{tab:zad6_f2}
\end{table}

\subsection{Interpretacja dla $f_2(x)$:}
\begin{itemize}
    \item \textbf{Metoda Bisekcji}: Zbiegła w 1 iteracji do dokładnego pierwiastka $r=0.0$.
    \item \textbf{Metoda Newtona} z $x_0=-1.0$: Zbiegła w 5 iteracjach do wartości bliskiej zeru.
    \item \textbf{Metoda Siecznych}: Zbiegła w 17 iteracjach, co jest wolniejsze ze względu na płaski przebieg funkcji w okolicy pierwiastka, utrudniający aproksymację pochodnej.
\end{itemize}

\subsection{Test dodatkowe dla metody Newton}
\begin{table}[h]
\centering
\begin{tabular}{@{} l l l l c c @{}}
\hline
Test & x0 & r & v & it & err \\
\hline
Newton (dla $f_1$) & $2$ & \texttt{0.9999999810061002} & \texttt{1.8993900008368314e-8} & 5 & 0 \\
Newton (dla $f_2$) & $2$ & \texttt{14.398662765680003} & \texttt{8.03641534421721e-6} & 10 & 0 \\
Newton (dla $f_2$) & $1$ & \texttt{1.0} & \texttt{0.36787944117144233} & 0 & 2 \\
\hline
\end{tabular}
\caption{Dodatkowe testy metody Newtona dla $f_1$ i $f_2$.}
\label{tab:newton_additional}
\end{table}

\subsection*{Interpretacja Testów Dodatkowych}

\begin{itemize}
    \item \textbf{Metoda Newtona dla $f_1(x)$ z $x_0=2$}: Metoda zbiegła (5 iteracji), co oznacza, że dla tej funkcji (gdzie pochodna $f_1'(x) = -e^{1-x} \neq 0$) wybór $x_0 > 1$ nie zakłócił drastycznie zbieżności.
    \item \textbf{Metoda Newtona dla $f_2(x)$ z $x_0=2$}: Metoda nie zbiegła do pierwiastka $r=0$, tylko do $r\approx14.4$ w 10 iteracjach.
    \item \textbf{Metoda Newtona dla $f_2(x)$ z $x_0=1$}: W tym punkcie pochodna $f_2'(x) = e^{-x}(1-x)$ jest równa $f_2'(1)=0$. Oznacza to, że warunek metody Newtona $f'(x) \neq 0$ jest naruszony. Test wykazał, że metoda \textbf{nie zbiegła} (err=2), co potwierdza, że wybór $x_0$ w punkcie zerowania się pochodnej prowadzi do problemów ze stabilnością i zbieżnością.
\end{itemize}

\subsection{Wnioski}
Na podstawie przeprowadzonych obliczeń i analizy wyników sformułowano następujące wnioski:
\begin{enumerate}
    \item \textbf{Pewność Zbieżności:} \textbf{Metoda Bisekcji} jest jedyną metodą gwarantującą zbieżność (pod warunkiem zmiany znaku) i jest najbardziej stabilna, niezależnie od zachowania pochodnej. 
    \item \textbf{Wrażliwość na $x_0$ (Metoda Newtona i Siecznych):}
    \begin{itemize}
        \item Wybór $x_0$ w pobliżu punktu, gdzie pochodna $f'(x)$ jest bliska zeru (np. $x_0=1$ dla $f_2(x)$), prowadzi do utraty zbieżności lub jej braku, ponieważ mianownik wzoru iteracyjnego dąży do zera. 
        \item Metoda Newtona może się "zapętlić" (przechodząc przez kilka punktów bez zbieżności) lub po kilku iteracjach wpaść w punkt, gdzie pochodna jest zerowa.
        \item Złe przybliżenie początkowe (np. $x_0=2$ dla $f_2(x)$) może prowadzić do zbieżności do innego miejsca zerowego lub nawet do rozbieżności. 
    \end{itemize}
    \item \textbf{Problemy Metody Siecznych:} Metoda Siecznych, wywodząc się z metody Newtona, może mieć te same problemy, np. otrzymana sieczna może mieć współczynnik kierunkowy równy 0 (pochodna bliska zeru) lub dążyć do zera w nieskończoności.
\end{enumerate}

\end{document}